\documentclass[11pt]{cernrep}
%\documentclass[12pt,letterpaper,twoside]{article}
%\documentclass[12pt,letterpaper]{article}
%\usepackage[pdftex]{graphicx,color}
\usepackage[pdftex]{graphicx,epsfig}
%\usepackage[dvips]{graphicx,color}
%\input{psfig}
%\input{epsf}
\usepackage{amsmath,amssymb}
%\usepackage[twoside,pdftex,letterpaper,text={6.5in,9in}]{geometry}
%\usepackage[twoside,dvips,letterpaper,text={6.5in,9in}]{geometry}
\usepackage[dvips,letterpaper,text={6.5in,9in}]{geometry}
\usepackage{fancyhdr}
\usepackage{verbatim}
\renewcommand{\baselinestretch}{1.1}
%\renewcommand{\theequation}{\thesection.\arabic{equation}}
%\numberwithin{equation}{section}

%       Symbol definitions
\newcommand\ltap{\
  \raise.3ex\hbox{$<$\kern-.75em\lower1ex\hbox{$\sim$}}\ }
\newcommand\gtap{\
  \raise.3ex\hbox{$>$\kern-.75em\lower1ex\hbox{$\sim$}}\ }
%%%%%%%%%%%%%%%%%%%%%%%%%%%%%%%%%%%%%%%%%%%%%%%%%%%%%%%%%%%%%%%%%%%%%%%%

%  \simge and \simle make "approx greater than" and "approx less than"
\newcommand\simge{\mathrel{%
   \rlap{\raise 0.511ex \hbox{$>$}}{\lower 0.511ex \hbox{$\sim$}}}}
\newcommand\simle{\mathrel{
   \rlap{\raise 0.511ex \hbox{$<$}}{\lower 0.511ex \hbox{$\sim$}}}}

%  \slashcar puts a slash through a character to represent contraction
%  with Dirac matrices. Use \not instead for negation of relations, and use
%  \hbar for hbar.
\newcommand{\slashchar}[1]%
        {\kern .25em\raise.18ex\hbox{$/$}\kern-.75em #1}
%%%%%%%%%%%%%%%%%%%%%%%%%%%%%%%%%%%%%%%%%%%%%%%%%%%%%%%%%%%%%%%%%%%%%%%%%%
\def\lsim{\mathrel{\raise.3ex\hbox{$<$\kern-.75em\lower1ex\hbox{$\sim$}}}}
\def\gsim{\mathrel{\raise.3ex\hbox{$>$\kern-.75em\lower1ex\hbox{$\sim$}}}}
%%%%%%%%%%%%%%%%%%%%%%%%%%%%%%%%%%%%%%%%%%%%%%%%%%%%%%%%%%%%%%%%%%%%%%%%%%
\newcommand{\bs}{\boldsymbol}

\newenvironment{changemargin}[2]{\begin{list}{}{
        \setlength{\topsep}{0pt}\setlength{\leftmargin}{0pt}
        \setlength{\rightmargin}{0pt}
        \setlength{\listparindent}{\parindent}
        \setlength{\itemindent}{\parindent}
        \setlength{\parsep}{0pt plus 1pt}
        \addtolength{\leftmargin}{#1}\addtolength{\rightmargin}{#2}
        }\item }{\end{list}}
%
\begin{document}
\title{
%\vskip -15mm
XXXXXXX}
\author{Benjamin Fuks$^1$, Jorge~de~Blais$^2$,
Kristin~Lohwasser$^3$, Pasquale Musella$^4$ and Ken~Mimasu$^5$}
\institute{$^1$XXX\\
$^2$XXX \\
$^3$Department of Physics and Astronomy, Sheffield University, Sheffield, UK \\
$^4$ } 
\maketitle

\begin{abstract}
 

\end{abstract}

%%%%%%%%%%%%%%%%%%%%%%%%%%%%%%%%%
%%%%%%%%%%%%%%%%%%%%%%%%%%%%%%%%%

\newpage

\section{Introduction}
\label{sec:intro}
% All


\section{Generated Models} 
\label{sec:SSSS}
%Ken
%%!TEX root = /Users/Ken/Work/Projects/LesHouches2017/STXSvsEFT/repo/LHStxsVsEft/Documentation/STXSvsSpecificEFTAna.tex
The production of a Higgs boson in association with an EW gauge boson can be considered one of the canonical LHC processes sensitive to SMEFT effects. Evidence for this process involving the $b\bar{b}$ decay mode of the Higgs and leptonic vector boson decays was finally observed in 2017~\cite{Aaboud:2017xsd,Sirunyan:2017elk}. Operators which modify the Higgs coupling to these gauge bosons, introducing momentum dependent interactions modify the production rate, especially enhancing it in the high energy regime. The associate production process can naturally access this region of phase space since the Higgs is produced recoiling against the associated vector, meaning that the $p_T$ of the Higgs or vector boson are a faithful proxy for the energy flowing through the EFT vertex. Of the many dimension-6 operators that can contribute to this process, we consider 
\begin{align}
    \mathcal{O}_{\sss HW} &= 
    \frac{ig}{\Lambda^2} \big[D_\mu \phi^\dag T_{2k} D_\nu \phi\big] 
    W^{k,\mu \nu},
\end{align}
an operator from the so called strongly interacting light Higgs (SILH) basis~\cite{Giudice:2007fh,Contino:2013kra}. Here, $T_{2k}$ refers to the generators of $SU(2)_L$ normalised such that $\{T_{2i},T_{2j}\}=\delta_{ij}/2$ and the covariant derivative, $D_\mu$, for the Higgs field is defined as
\begin{align}
    D_\mu\phi &= \partial_\mu \phi -  i g T_{2k} W_\mu^k \phi - \frac12 i g' B_\mu \phi,
\end{align}
with $g$ and $g^\prime$ the weak and hypercharge gauge couplings respectively.

A global fit~\cite{Ellis:2014jta} combining information from precision measurements at LEP and LHC Run 1 data constrains the Wilson coefficient to lie in the range 
\begin{align}
    \frac{m_{\sss W}^2}{\Lambda^2}c_{\sss HW} = [-0.07,\,0.03].
\end{align}
A sensitivity estimate for LHC run 2 was also performed in~\cite{Degrande:2016dqg} by projecting an 8 TeV ZH analysis~\cite{TheATLAScollaboration:2013lia} to 13 TeV, a single overflow bin of the $Z$ $p_T$ distribution, indicating that factor of a few \red{Percent? Apples? Pears? Order of magnitude?} improvements could be made even with this limited information. This indicates that the STXS measurements are likely to provide even greater sensitivity to this parameter.

Motivated by the limits from the global fit, we select the following benchmark values for $c_{\sss HW}$:
\begin{align}
    c_{\sss HW} = \pm 0.03 \text{ and } \pm0.01\,,
\end{align}
setting, for simplicity, $\Lambda^2=m_{\sss W}^2$. The first value roughly saturates the positive end of the limit, and has been shown to yield drastic effects in the kinematic tails of distributions~\cite{Degrande:2016dqg}. The second corresponds to a smaller, yet potentially accessible value of the parameter that may better test the relative discriminating power of relatively small effects between the two methods we investigate. We simulate our Monte Carlo samples at leading order using {\sc MadGraph5\_aMC@NLO}~\cite{Alwall:2014hca} with the public {\sc HELatNLO}~\cite{Degrande:2016dqg,Alloul:2013naa} {\sc FeynRules}~\cite{Alloul:2013bka} model, exploiting reweighting methods to simulate multiple parameter space points (including the SM) simultaneously. We also include the dominant irreducible background contribution from $Z + b\bar{b}$ production with the $Z$-boson decaying leptonically. Showering and hadronisation, as well as the Higgs boson decay to $b\bar{b}$ are performed with {\sc PYTHIA8}~\cite{pythia8} and the events are reconstructed from hadron level with {\sc MadAnalysis5}~\cite{Conte:2012fm} which makes use of {\sc FastJet}~\cite{Cacciari:2011ma}.



\section{Analysis}
\label{sec:tools}
%%!TEX root = /Users/Ken/Work/Projects/LesHouches2017/STXSvsEFT/repo/LHStxsVsEft/Documentation/STXSvsSpecificEFTAna.tex
To begin the analysis on a realistic level, we first perform a simple fiducial selection on the event samples, to emulate a typical LHC selection that would be performed for the ZH process. To this end, we also implement a $p_T$ and $|\eta|$ dependent smearing function on the $b$-jet momenta to approximate finite detector resolution effects following the parametrised functions determined by the CMS particle-flow performance analysis~\cite{Sirunyan:2017ulk}.

Events are required to have two leptons satisfying $p_T > 25$ GeV and $|\eta|< 2.5$. Exactly two $b$-jets, as identified using truth-level information by {\sc MadAnalysis5}, are required satisfying $p_T > 20$ GeV and $|\eta|< 2.5$. We assume a flat $b$-tagging efficiency of $70\%$, corresponding to the DeepCSV medium working point defined in~\cite{Sirunyan:2017ezt}. Additionally, $Z$- and Higgs-boson mass windows are imposed on the invariant masses of the lepton and $b$-jet pairs such that $75 < M_{\ell\ell} < 105$ GeV and  $60 < M_{\ell\ell} < 140$ GeV. This defines our fiducial volume on which both the BDT training and STXS binning will be performed. Table~\ref{tab:FiducialXS} summarises the cross sections obtained after the fiducial selection for the Monte Carlo samples generated. The $H\to b\bar{b}$ branching fraction is computed in the SM and the SMEFT benchmark points using the e{\sc HDECAY}~\cite{Contino:2014aaa} interface of {\sc Rosetta}~\cite{Falkowski:2015wza} and folded into the cross section results.
\begin{table}
    \centering
    \begin{tabular}{|l|l|}
        \hline
        $pp\to b\,\bar{b}\,\ell^+\,\ell^-$& $\sigma_{\text{fid.}} [fb]$\tabularnewline
        \hline
        $ZH$ SM&5.55\tabularnewline
        $ZH$ $c_{\sss HW} = 0.03$&3.64\tabularnewline
        $ZH$ $c_{\sss HW} = -0.03$&2.21\tabularnewline
        $ZH$ $c_{\sss HW} = 0.01$&3.38\tabularnewline
        $ZH$ $c_{\sss HW} = -0.01$&2.50\tabularnewline
        $Z b\bar{b}$ SM&291.3\tabularnewline
        \hline
    \end{tabular}
    \caption{\label{tab:FiducialXS} Cross sections obtained at LO after imposing the fiducial selection cuts described in Section\ref{sec:tools}.}
\end{table}
Clearly, the $Z b\bar{b}$ background is overwhelmingly large even after the Higgs mass window selection. A realistic analysis will employ data driven methods to constrain this background using control regions. In the next section, one part of the BDT training is used to optimally reject this background in favour of the SM $ZH$ process.

%  


\section{Statistical Hypothesis testing}
\label{sec:test}

%All!
%\input{analysis.tex}


\section{Conclusions}
\label{sec:conclusions}



\section*{Acknowledgments}

We thank the organizers and conveners of the Les Houches workshop, ``Physics
at TeV Colliders'', for a stimulating meeting. This work has received funding from the European Union's Horizon 2020 research and innovation programme / ERC Grant Agreement n. 715871.


%\vfil\eject


\bibliography{STXSvsSpecificEFTAna}
\bibliographystyle{utcaps}
%\bibliographystyle{lesHouches}
\end{document}
