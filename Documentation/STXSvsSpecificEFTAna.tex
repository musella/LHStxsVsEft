\documentclass[11pt]{cernrep}
%\documentclass[12pt,letterpaper,twoside]{article}
%\documentclass[12pt,letterpaper]{article}
%\usepackage[pdftex]{graphicx,color}
\usepackage[pdftex]{graphicx,epsfig}
%\usepackage[dvips]{graphicx,color}
%\input{psfig}
%\input{epsf}
\usepackage{amsmath,amssymb}
%\usepackage[twoside,pdftex,letterpaper,text={6.5in,9in}]{geometry}
%\usepackage[twoside,dvips,letterpaper,text={6.5in,9in}]{geometry}
\usepackage[dvips,letterpaper,text={6.5in,9in}]{geometry}
\usepackage{fancyhdr}
\usepackage{verbatim}
\renewcommand{\baselinestretch}{1.1}
%\renewcommand{\theequation}{\thesection.\arabic{equation}}
%\numberwithin{equation}{section}

%       Symbol definitions
\newcommand\ltap{\
  \raise.3ex\hbox{$<$\kern-.75em\lower1ex\hbox{$\sim$}}\ }
\newcommand\gtap{\
  \raise.3ex\hbox{$>$\kern-.75em\lower1ex\hbox{$\sim$}}\ }
%%%%%%%%%%%%%%%%%%%%%%%%%%%%%%%%%%%%%%%%%%%%%%%%%%%%%%%%%%%%%%%%%%%%%%%%

%  \simge and \simle make "approx greater than" and "approx less than"
\newcommand\simge{\mathrel{%
   \rlap{\raise 0.511ex \hbox{$>$}}{\lower 0.511ex \hbox{$\sim$}}}}
\newcommand\simle{\mathrel{
   \rlap{\raise 0.511ex \hbox{$<$}}{\lower 0.511ex \hbox{$\sim$}}}}

%  \slashcar puts a slash through a character to represent contraction
%  with Dirac matrices. Use \not instead for negation of relations, and use
%  \hbar for hbar.
\newcommand{\slashchar}[1]%
        {\kern .25em\raise.18ex\hbox{$/$}\kern-.75em #1}
%%%%%%%%%%%%%%%%%%%%%%%%%%%%%%%%%%%%%%%%%%%%%%%%%%%%%%%%%%%%%%%%%%%%%%%%%%
\def\lsim{\mathrel{\raise.3ex\hbox{$<$\kern-.75em\lower1ex\hbox{$\sim$}}}}
\def\gsim{\mathrel{\raise.3ex\hbox{$>$\kern-.75em\lower1ex\hbox{$\sim$}}}}
%%%%%%%%%%%%%%%%%%%%%%%%%%%%%%%%%%%%%%%%%%%%%%%%%%%%%%%%%%%%%%%%%%%%%%%%%%
\newcommand{\bs}{\boldsymbol}

\newenvironment{changemargin}[2]{\begin{list}{}{
        \setlength{\topsep}{0pt}\setlength{\leftmargin}{0pt}
        \setlength{\rightmargin}{0pt}
        \setlength{\listparindent}{\parindent}
        \setlength{\itemindent}{\parindent}
        \setlength{\parsep}{0pt plus 1pt}
        \addtolength{\leftmargin}{#1}\addtolength{\rightmargin}{#2}
        }\item }{\end{list}}
%
\begin{document}
\title{
%\vskip -15mm
XXXXXXX}
\author{Jorge~de~Blas$^1$,
Kristin~Lohwasser$^2$, Pasquale~Musella$^3$ and Ken~Mimasu$^4$}
\institute{$^1$XXX \\
$^2$Department of Physics and Astronomy, Sheffield University, Sheffield, UK \\
$^3$XXX\\
$^4$Centre for Cosmology, Particle Physics and Phenomenology (CP3), Universit\'e
catholique de Louvain, Chemin du Cyclotron, 2, B-1348 Louvain-la-Neuve, Belgium } 
\maketitle

\begin{abstract}
 

\end{abstract}

%%%%%%%%%%%%%%%%%%%%%%%%%%%%%%%%%
%%%%%%%%%%%%%%%%%%%%%%%%%%%%%%%%%

\newpage

\section{Introduction}
\label{sec:intro}
% All
The Standard Model Effective Field Theory (SMEFT) is, by now, a well established framework for parametrising new physics effects in the interactions of Standard Model (SM) particles in a model independent way. It has been and continues to be a key part of the LHC programme, complementary to direct searches for new physics. The framework employs an operator expansion in canonical dimension suppressed by a generic cutoff scale, $\Lambda$, assumed to be much larger than the electroweak (EW) scale, situated by the vacuum expectation value of the Higgs field, $v$. The SM Lagrangian is thus supplemented by higher dimension operators and truncated at dimension-6, the lowest dimension at which $B$- and $L$-number conserving operators appear.
\begin{align}
    \mathcal{L}=\mathcal{L}_{SM}+\sum_i \frac{c_i}{\Lambda^2}\mathcal{O}^i+\cdots,
\end{align} 
Where the dimension-5 Weinberg operator for neutrino masses has been neglected. New physics effects are then expected to appear scaling between $\sim v^2/\Lambda^2$ and a heightened energy dependence of $\sim E^2/\Lambda^2$.

One of the main strengths of the LHC in this respect is its ability to probe the high energy regime, in which it is expected that the sensitivity to EFT effects will be maximised to the large momentum flow through the modified  vertices accessing the stronger energy growth of the higher dimensional operators. Furthermore, the discovery of the Higgs boson in 2012 has opened a brand new avenue in constraining the SMEFT parameter space consisting of the various operators involving Higgs fields. Measurements of Higgs production and decay modes have already provided new constraints on many operators and have also helped to constrain some blind directions in existing fits to low-energy data such as precision electroweak measurements at LEP. 

In the first run of the LHC, a very successful programme of signal strength measurements took place, in which information from many searches was combined into a global fit to overall coupling modifiers between the Higgs and the rest of the SM particles. The natural evolution of these measurements for Run 2 is to subdivide the phase space and work towards differential observables in Higgs production and decay. To this end, a staged approach termed Simplified Template Cross Sections (STXS) is being developed~\cite{deFlorian:2016spz}, consisting of an increasingly fine-grained binning of kinematic observables, separated by production and decay mode. The aim is to provide measurements in mutually exclusive regions of phase space, performed in simplified fiducial volumes and unfolded to remove detector and acceptance effects. Ideally, these will be designed to maximise sensitivity to new physics.

Being one of the main elements of LHC searches for non-SM physics, it is of great interest to evaluate the sensitivity of the STXS measurements to SMEFT effects in Higgs boson interactions, particularly since they will be able to access these high energy tails of kinematic distributions. In particular, one would like to know how the information provided by a generic framework such as the STXS would compare to an optimised, dedicated search for SMEFT effects. Naively, one may expect some loss of information given, \emph{e.g.}, the finite binning of the distributions. In this study, we aim to quantify this difference by comparing and contrasting the ability to constrain SMEFT effects in Higgs production between the STXS measurements and an optimised analysis making use of multivariate methods to extract the maximum classification power of the SMEFT signals. We consider the concrete scenario of the (ZH) production of a Higgs boson decaying into a pair of $b$-quarks in association with a $Z$-boson  decaying to a pair of leptons, in the presence of a single EFT operator. We simulate several benchmark values for the operator Wilson coefficient within existing constraints from global fits along with the dominant reducible SM background and evaluate the statistical discriminating power of a hypothesis test using the STXS measurements versus a multivariate Boosted Decision Tree (BDT) classifier.

The paper is organised as follows. We first describe the Monte Carlo event generation procedure for the SM and EFT benchmarks in section~\ref{sec:gen}. Section~\ref{sec:tools} first describes the fiducial selection employed, the training and analysis implemented using the BDT classifier and the STXS binning used for ZH. In Section~\ref{sec:test}, we summarise the results of the selections and binnings and perform a statistical hypothesis test to quantify the relative strengths of the two methods before concluding and laying out the avenues for further investigation in Section~\ref{sec:conclusions}

\section{Generated Models} 
\label{sec:gen}
%Ken
%!TEX root = /Users/Ken/Work/Projects/LesHouches2017/STXSvsEFT/repo/LHStxsVsEft/Documentation/STXSvsSpecificEFTAna.tex
The production of a Higgs boson in association with an EW gauge boson can be considered one of the canonical LHC processes sensitive to SMEFT effects. Evidence for this process involving the $b\bar{b}$ decay mode of the Higgs and leptonic vector boson decays was finally observed in 2017~\cite{Aaboud:2017xsd,Sirunyan:2017elk}. Operators which modify the Higgs coupling to these gauge bosons, introducing momentum dependent interactions modify the production rate, especially enhancing it in the high energy regime. The associate production process can naturally access this region of phase space since the Higgs is produced recoiling against the associated vector, meaning that the $p_T$ of the Higgs or vector boson are a faithful proxy for the energy flowing through the EFT vertex. Of the many dimension-6 operators that can contribute to this process, we consider 
\begin{align}
    \mathcal{O}_{\sss HW} &= 
    \frac{ig}{\Lambda^2} \big[D_\mu \phi^\dag T_{2k} D_\nu \phi\big] 
    W^{k,\mu \nu},
\end{align}
an operator from the so called strongly interacting light Higgs (SILH) basis~\cite{Giudice:2007fh,Contino:2013kra}. Here, $T_{2k}$ refers to the generators of $SU(2)_L$ normalised such that $\{T_{2i},T_{2j}\}=\delta_{ij}/2$ and the covariant derivative, $D_\mu$, for the Higgs field is defined as
\begin{align}
    D_\mu\phi &= \partial_\mu \phi -  i g T_{2k} W_\mu^k \phi - \frac12 i g' B_\mu \phi,
\end{align}
with $g$ and $g^\prime$ the weak and hypercharge gauge couplings respectively.

A global fit~\cite{Ellis:2014jta} combining information from precision measurements at LEP and LHC Run 1 data constrains the Wilson coefficient to lie in the range 
\begin{align}
    \frac{m_{\sss W}^2}{\Lambda^2}c_{\sss HW} = [-0.07,\,0.03].
\end{align}
A sensitivity estimate for LHC run 2 was also performed in~\cite{Degrande:2016dqg} by projecting an 8 TeV ZH analysis~\cite{TheATLAScollaboration:2013lia} to 13 TeV, a single overflow bin of the $Z$ $p_T$ distribution, indicating that factor of a few \red{Percent? Apples? Pears? Order of magnitude?} improvements could be made even with this limited information. This indicates that the STXS measurements are likely to provide even greater sensitivity to this parameter.

Motivated by the limits from the global fit, we select the following benchmark values for $c_{\sss HW}$:
\begin{align}
    c_{\sss HW} = \pm 0.03 \text{ and } \pm0.01\,,
\end{align}
setting, for simplicity, $\Lambda^2=m_{\sss W}^2$. The first value roughly saturates the positive end of the limit, and has been shown to yield drastic effects in the kinematic tails of distributions~\cite{Degrande:2016dqg}. The second corresponds to a smaller, yet potentially accessible value of the parameter that may better test the relative discriminating power of relatively small effects between the two methods we investigate. We simulate our Monte Carlo samples at leading order using {\sc MadGraph5\_aMC@NLO}~\cite{Alwall:2014hca} with the public {\sc HELatNLO}~\cite{Degrande:2016dqg,Alloul:2013naa} {\sc FeynRules}~\cite{Alloul:2013bka} model, exploiting reweighting methods to simulate multiple parameter space points (including the SM) simultaneously. We also include the dominant irreducible background contribution from $Z + b\bar{b}$ production with the $Z$-boson decaying leptonically. Showering and hadronisation, as well as the Higgs boson decay to $b\bar{b}$ are performed with {\sc PYTHIA8}~\cite{pythia8} and the events are reconstructed from hadron level with {\sc MadAnalysis5}~\cite{Conte:2012fm} which makes use of {\sc FastJet}~\cite{Cacciari:2011ma}.



\section{Analysis}
\label{sec:tools}
%!TEX root = /Users/Ken/Work/Projects/LesHouches2017/STXSvsEFT/repo/LHStxsVsEft/Documentation/STXSvsSpecificEFTAna.tex
To begin the analysis on a realistic level, we first perform a simple fiducial selection on the event samples, to emulate a typical LHC selection that would be performed for the ZH process. To this end, we also implement a $p_T$ and $|\eta|$ dependent smearing function on the $b$-jet momenta to approximate finite detector resolution effects following the parametrised functions determined by the CMS particle-flow performance analysis~\cite{Sirunyan:2017ulk}.

Events are required to have two leptons satisfying $p_T > 25$ GeV and $|\eta|< 2.5$. Exactly two $b$-jets, as identified using truth-level information by {\sc MadAnalysis5}, are required satisfying $p_T > 20$ GeV and $|\eta|< 2.5$. We assume a flat $b$-tagging efficiency of $70\%$, corresponding to the DeepCSV medium working point defined in~\cite{Sirunyan:2017ezt}. Additionally, $Z$- and Higgs-boson mass windows are imposed on the invariant masses of the lepton and $b$-jet pairs such that $75 < M_{\ell\ell} < 105$ GeV and  $60 < M_{\ell\ell} < 140$ GeV. This defines our fiducial volume on which both the BDT training and STXS binning will be performed. Table~\ref{tab:FiducialXS} summarises the cross sections obtained after the fiducial selection for the Monte Carlo samples generated. The $H\to b\bar{b}$ branching fraction is computed in the SM and the SMEFT benchmark points using the e{\sc HDECAY}~\cite{Contino:2014aaa} interface of {\sc Rosetta}~\cite{Falkowski:2015wza} and folded into the cross section results.
\begin{table}
    \centering
    \begin{tabular}{|l|l|}
        \hline
        $pp\to b\,\bar{b}\,\ell^+\,\ell^-$& $\sigma_{\text{fid.}} [fb]$\tabularnewline
        \hline
        $ZH$ SM&5.55\tabularnewline
        $ZH$ $c_{\sss HW} = 0.03$&3.64\tabularnewline
        $ZH$ $c_{\sss HW} = -0.03$&2.21\tabularnewline
        $ZH$ $c_{\sss HW} = 0.01$&3.38\tabularnewline
        $ZH$ $c_{\sss HW} = -0.01$&2.50\tabularnewline
        $Z b\bar{b}$ SM&291.3\tabularnewline
        \hline
    \end{tabular}
    \caption{\label{tab:FiducialXS} Cross sections obtained at LO after imposing the fiducial selection cuts described in Section\ref{sec:tools}.}
\end{table}
Clearly, the $Z b\bar{b}$ background is overwhelmingly large even after the Higgs mass window selection. A realistic analysis will employ data driven methods to constrain this background using control regions. In the next section, one part of the BDT training is used to optimally reject this background in favour of the SM $ZH$ process.

%  


\section{Statistical Hypothesis testing}
\label{sec:test}

%All!
%!TEX root = /Users/Ken/Work/Projects/LesHouches2017/STXSvsEFT/repo/LHStxsVsEft/Documentation/STXSvsSpecificEFTAna.tex
A statistical analysis is carried out to estimate the sensitivity to SMEFT effects in Higgs boson interactions of the STXS measurements compared to that of a dedicated analysis using a multivariate classifier. To this effect, a simple significance analysis is used based on the {\sc RooStats} framework~\cite{Moneta:2010pm} which determines the expected signifiance using an asymptotic calculator with nominal Asimov data sets and a one-sided profile likelihood. 

Naturally, a few approximations have been made. No systematic uncertainties have been considered, even if the generated events have been smeared to reflect the limited resolution and corrected for the finite efficiencies of $b$-tagging. The measurement in the STXS bins requires an extrapolation from the measured phase space, which includes those selections applied specifically to reject backgrounds, in this case from the $Z b\bar{b}$ process. In case of the present analysis this is achieved using a BDT specifically trained to select $H\to b\bar{b}$ over $Z b\bar{b}$ events. For the STXS analysis, a BDT cut of $<$0.1 is applied, retaining 18.6\% of all SM Higgs events but less than 1\% of the $Z b\bar{b}$ background. For the BDT analysis targeting EFT operators, two settings are explored: Using the same BDT requirement as in the STXS analysis and additionally using a selection requirement of the BDT of $<$0.055 (and 0.06 for $c_{\sss HW} <0$) which leads to similar acceptances as in the STXS case. 

In a real-life analysis, the effects of using the BDT-selection would have to be accounted for. Herein, it is assumed, that these are perfectly known for both SM and BSM events with $c_{\sss HW} \neq 0$. Different acceptances of SM and BSM however can play a role, when testing for BSM physics in the STXS phase space, to which the events selected on detector level have been extrapolated assuming the SM acceptances alone. For information, the acceptances of the BDT-selection for both, SM and BSM events with $c_{\sss HW} \neq 0$ are summarized in Table~\ref{tab:bkg_acceptance}. The same applies to a dedicated SMEFT analysis as presented here: the acceptance of the first BDT selection needs to be modelled well as also the BDT classifier distribution used to constrain the BSM SMEFT phase space \red{this sentence is a bit unclear...}. The acceptance of the first BDT selection is summarized in Table~\ref{tab:bkg_acceptance}. The acceptance for the SM Higgs can (depending on the BDT cut) be indeed very similar for both STXS BDT and EFT optimized-BDT (around 18\%). The acceptance for events with a Wilson coefficent $c_{\sss HW}=0.03$ is larger than that by about a factor of 1.5 whilst it is smaller by the 25\% for $c_{\sss HW}=-0.03$. For the samples produced with a smaller Wilson coefficient, $c_{\sss HW} = \pm0.01$, the acceptances are slightly closer to the SM Higgs scenario, which is expected since for $c_{\sss HW} \varinjlim 0$ the SM is restored. The smaller the Wilson coefficients, the smaller issues from acceptance effects. 

\begin{table}[!h]
\begin{center}
{\scriptsize
\begin{tabular}{|l|c|c|c|}
\hline  
Sample		&  STXS: Acceptance (BDT$_{\mathrm{SM}}$) [\%] &  BDT: Acceptance (BDT$_{\mathrm{SM}}$, &  BDT: Acceptance (BDT$_{\mathrm{SM}}$, 	\\ 
		&  &  same as STXS SM-acceptance) [\%] &  same cut as STXS-BDT) [\%] 	\\ \hline

Higgs & 	18.6	&  18.5	& 29.6	\\
Higgs & 	18.6	&  18.3	& 33.1	\\
Higgs & 	18.6	&  18.6	& 33.4	\\ 
Higgs & 	18.6	&  19.8	& 34.2	\\ \hline

$c_{\sss HW} = 0.03$ &	31.1 & 31.8& 42.7\\
$c_{\sss HW} = -0.03$ &	14.4 &13.4& 28.2\\

$c_{\sss HW} = 0.01$ &	22.9	&23.0& 37.9\\
$c_{\sss HW} = -0.01$ &	15.9	&17.2& 31.7\\ \hline

\end{tabular}
}
\vskip0.5truecm
\caption{Acceptances (in \%) of the first BDT selection, meant to separate $H\to b\bar{b}$ from $Z b\bar{b}$ production.}
\label{tab:bkg_acceptance}
\end{center}
\end{table} 

After application of the BDT requirements, either the STXS binning is applied or the distribution of a second BDT \red{(the second BDT output? this sounds like a separate thing was trained after cutting on the first)} trained to enhance events with $c_{\sss HW} \neq 0$ is used to estimate the sensitivity to new physics as expressed by the Wilson coefficients. Three different luminosity scenarios are investigated: the full Run-2 results, corresponding to 150 fb$^{-1}$, the integrated luminosity projected for LHC Run-3 (300 fb$^{-1}$) and the expected data collected at the High-Luminosity LHC (3000 fb$^{-1}$). The significance is determined simultaneously in the 6 STXS bins (see Section~\ref{sec:stxs}) and in the BDT discriminant (10 bins with sizeable entries, with more bins not changing the significance and less bins reducing it slightly). For both, STXS and BDT discriminant, no uncertainties on the shapes of these distributions are assumed. Figure~\ref{fig:hypotest} depicts the distributions used as inputs in the SXTS (left) and the BDT (right) case for 300 fb$^{-1}$. The SM hypothesis ($Z b\bar{b}$ + $H\to b\bar{b}$) is shown as blue line, whereas the BSM signal with $c_{\sss HW} = 0.03$ is shown as red line. They are added in the signal+background hypothesis which is depicted as dashed black line. A simple significance test for the signal+background hypothesis is carried out using the {\sc RooStats} framework~\cite{Moneta:2010pm} for these binned distributions. 
 
 
\begin{figure}[htb]
\centering
      \includegraphics[width=0.45\textwidth]{plots/HypoTest_STXS.pdf}
      \includegraphics[width=0.45\textwidth]{plots/HypoTest_BDT.pdf}
      \caption{Predicted number of events selected for 300 fb$^{-1}$. Since the acceptance extrapolation into the STXS phase space does not change the available data statistics, it it not applied here, i.e. the distributions are shown after the first cut on the BDT classifier to reject the $Z b\bar{b}$ background.}
\label{fig:hypotest}
\end{figure}

To get a feeling of how realistic the scenario investigated herein is, the significance of a Higgs discovery in the STXS scenario is also investigated in addition to investigating the BSM sensitivity. The expected significances for the 2-lepton channel investigated here are 1.9 for ATLAS~\cite{Aaboud:2017xsd} and 1.8 for CMS~\cite{Sirunyan:2017elk} for $\sim$36 fb$^{-1}$. This is about what is expected from the simplified studies herein, which do not account for systematic uncertainties (which make up half the total uncertainty in the measurements) but are not optimized for a Higgs observation. The expected significance of a Higgs signal compared to a background (i.e. $Z b\bar{b}$) only sample is shown in Table~\ref{tab:significances}.

Table~\ref{tab:significances} also summarizes the significances found for the three luminosity scenarios for the hypothesis tests for the STXS and the BDT approach for Wilson coefficients of $c_{\sss HW} = \pm0.03$ and $c_{\sss HW} = \pm0.01$. In the case of the BDT approach and $c_{\sss HW} = \pm0.03$, three alternatives were tested. For these either the first BDT selection with the same STXS SM-acceptance or the same BDT cut as STXS are investigated and in addition a non-optimal BDT discriminant is used, that was trained not on the targeted Wilson coefficient but on it's opposite charge equivalent. 

Surprisingly, the BDT approach that dedicatedly targets the BSM encapsulated by the Wilson coefficient does perform rather similarly to the STXS, independently of which $Z b\bar{b}$ rejection setting was used. As expected, the usage of the non-optimal BDT discriminant reduces the expected significance significantly. This can be understood from the fact that $ZH$ production at LO, being a $2\to2$ process, has only two kinematic degrees of freedom. Large kinematic effects from the EFT may be so striking in this case that the BDT is unable to provide significantly more separation power. However, for the smaller Wilson coefficient of $c_{\sss HW} = \pm0.01$, the BDT performs slightly better compared to the STXS, this difference seemingly increases as the deviation from SM predictions decreases as the greatest disparity in performance exists in the $c_{\sss HW} = -0.01$ benchmark, which differs the least from the SM. It appears that the BDT is better able to provide discrimination in the case where the new physics contribution is more evenly distributed across the phase space, as opposed to the extreme cases where the high $p_T$ region is dominated by new physics. This however would need to better studied. Further investigations concerning a comparison between BDT and STXS for less well-defined kinematic environments would be also interesting (\red{move this to conclusion & outlook}).


 %c_{\sss HW} = \pm 0.03 \text{ and } \pm0.01\,,

\begin{table}[!h]
\begin{center}
{\scriptsize
\begin{tabular}{|l|c|c|c|c|c|c|c|c|c|}
\hline  
Hypothesis test		&  Full Run-2 (150 fb$^{-1}$) 	& LHC Run-3 (300 fb$^{-1}$) 	& HL-LHC  (3000 fb$^{-1}$) \\ \hline

STXS: Higgs discovery	& 		3.01 &  	3.70
			& 8.06 \\ \hline

STXS: $c_{\sss HW} = 0.03$ &		6.44 	&	8.82	& 26.46\\
STXS: $c_{\sss HW} = -0.03$ &		1.66&	2.24 &  257 7.19\\
%%------------------------------
BDT: $c_{\sss HW} = 0.03$ (STXS SM-acceptance)&		 6.29&	8.58 & 25.61\\
BDT: $c_{\sss HW} = -0.03$ (STXS SM-acceptance)&		1.80&	2.44& 7.24\\ \hline
%%------------------------------
BDT: $c_{\sss HW} = 0.03$ (same BDT cut as STXS)&	6.17&	8.64 & 27.03\\
BDT: $c_{\sss HW} = -0.03$ (same BDT cut as STXS)&	1.74 &	2.08 & 7.50\\ \hline
%%------------------------------
BDT: $c_{\sss HW} = 0.03$ (alt BDT cut)		&4.40	&	6.15	&19.18\\
BDT: $c_{\sss HW} = -0.03$ (alt BDT cut)	&1.44	&	2.41	& 6.69\\ \hline
%%------------------------------
STXS: $c_{\sss HW} = 0.01$ &		2.26		&	3.04			& 8.78\\
STXS: $c_{\sss HW} = -0.01$ &		1.08		&	1.46			& 4.30\\ 
%%------------------------------
BDT: $c_{\sss HW} = 0.01$ &		2.62		&	3.07			& 8.90\\
BDT: $c_{\sss HW} = -0.01$ &		1.44		&	1.99			& 6.10\\ \hline
\end{tabular}
}
\vskip0.5truecm
\caption{Expected significances for different scenarios.}
\label{tab:significances}
\end{center}
\end{table} 



\section{Conclusions}
\label{sec:conclusions}
%!TEX root = /Users/Ken/Work/Projects/LesHouches2017/STXSvsEFT/repo/LHStxsVsEft/Documentation/STXSvsSpecificEFTAna.tex



\section*{Acknowledgments}

We thank the organizers and conveners of the Les Houches workshop, ``Physics
at TeV Colliders'', for a stimulating meeting. This work has received funding from the European Union's Horizon 2020 research and innovation programme / ERC Grant Agreement n. 715871. K. M. is supported in part by the Belgian Federal Science Policy Office through the Interuniversity Attraction Pole P7/37 and by the European Union’s Horizon 2020 research and innovation programme under the Marie Sk\l{}odowska-Curie grant agreement No. 707983.


%\vfil\eject


\bibliography{STXSvsSpecificEFTAna}
\bibliographystyle{utcaps}
%\bibliographystyle{lesHouches}
\end{document}
