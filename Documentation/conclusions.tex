%!TEX root = /Users/Ken/Work/Projects/LesHouches2017/STXSvsEFT/repo/LHStxsVsEft/Documentation/STXSvsSpecificEFTAna.tex
We have performed an exploratory study comparing the sensitivity to higher dimensional operators of the proposed STXS measurements in $ZH$ production to an optimised analysis exploiting multivariate methods. We considered four benchmark scenarios in which the $\mathcal{O}_{\sss HW}$ operator coefficient is set to values $c_{HW}=\pm 0.03$ and $c_{HW}=\pm 0.01$. The former case corresponds to saturating existing limits from a global fit to LHC Run 1 data and precision electroweak measurements, while the latter case intends to showcase a scenario with smaller deviations from SM expectations. The sensitivity was quantified by the expected statistical significance against the SM hypothesis obtainable after a collected integrated luminosity of 150, 300 and 3000 fb$^{-1}$, taking into account the presence of the dominant SM background of $Zb\bar{b}$ production. The contribution of this background is efficiently mitigated by training a BDT classifier to distinguish this process from SM $ZH$ production and first cutting on this discriminant before performing the two alternative SM vs EFT significance analyses. The discriminant was able to effectively reduce this background contribution down by two orders of magnitude.

Overall, very large significances can be expected for the benchmarks saturating the current limits, while the benchmarks for the smaller Wilson coefficients are not likely to be identified beyond 3$\sigma$ until the High-Luminosity LHC run. We observe that the final performance of the BDT analysis does not differ significantly from the differential information in $Z$-boson $p_T$ offered by the STXS, with the exception of the $c_{\sss HW}=-0.01$ case, which predicts the smallest deviation from the SM case. Here, the discriminating power of the BDT output over the differential $p_T$ distributions becomes apparent, suggesting that once the sensitivity of the STXS measurements is saturated, moving towards optimised multivariate methods remains well-motivated. The exercise was performed in a simplified situation, largely ignoring detector effects besides a parametrised $b$-jet smearing implementation and $b$-tagging efficiency corrections as well as all other potential sources of systematic uncertainty. We leave a more thorough investigation, including these effects as well as the possibility of including other significant backgrounds to a follow-up study. 

By comparing the optimised BDT discriminants for the different EFT benchmarks, we conclude that there is significant information overlap between them but that some parameter dependence remains. This means that one would benefit from a parametrised learning approach, in which the new physics parameter is also fed in as an input to the discriminant training. This can be understood from the presence of both an interference and squared contribution of the EFT $ZH$ amplitude in the new physics signal. The shape of EFT squared contribution has the benefit of being independent of the value of the Wilson coefficient, while the relative impact of the interference term depends very much on this value. In the `large' $c_{\sss HW}$ benchmarks, the contribution from the quadratic term in the Wilson coefficient is clearly dominant at high energies, as evidenced by the positive relative contribution over the SM prediction for both $\pm0.03$ in the STXS overflow bin, see Figure~\ref{fig:stxs_crosssec}. On the other hand the $-0.01$ case consistently predicts a deficit with respect to the SM. The fact that the BDT outperforms for this benchmarks may imply that one can obtain better sensitivity using these methods for EFT signals in which the interference with the SM amplitude is significant, which may also be considered more `well-behaved' concerning the EFT expansion. However, this effect may also be caused by the greater loss in acceptance post $Zb\bar{b}$ BDT cut suffered by the negative $c_{\sss HW}$ benchmarks and should be further investigated.

One should bear in mind that in this first study, a rather kinematically simple process has been chosen.
Indeed, in $ZH$ production the $p_T$ $Z$-boson is strongly correlated to the energy flow through the production vertex, in which the EFT effects occur. It is therefore not surprising that we do not observe a huge difference in significance in this case. Further investigations concerning a comparison between BDT and STXS for more complicated kinematic environments would be interesting, \emph{e.g.}, for other $2\to3$ production modes such as vector boson fusion or $t\bar{t}H$ associated production.  Furthermore, it should be noted that although our BDT analysis is touted as an `optimised' discriminating method, the fully potential of the BDT information was not exploited in this analysis. In short, by cutting on the SM vs $Zb\bar{b}$ variable and fitting on the resulting one-dimensional discriminant, some amount of exclusion power was sacrificed for the sake of simplicity. In the ideal case, a two-dimensional fit on the initial BDT classifier would be performed, an exercise which we leave for the follow-up study.
%\red{I would add a short discussion of the acceptance effects that are usually not accounted for.}
